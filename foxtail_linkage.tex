\documentclass[11pt]{article}
\pdfoutput=1
\usepackage{simplemargins}
\usepackage{url}
\usepackage[pdftex]{graphicx}
\graphicspath{{figures/}}
\usepackage{siunitx}
\bibliographystyle{plain}
\setlength{\parindent}{0pt} 
\setlength{\parskip}{1.6ex} 
\setallmargins{1in} 
\linespread{1.6}
\usepackage[backend=biber,
             natbib=true,
             style=authoryear,
             bibstyle=science,
             maxbibnames=50,
             maxcitenames=3,
             sorting=nty]{biblatex}
\addbibresource{refs.bib}
\begin{document}


% Title must be 150 characters or less
\begin{flushleft} 
{\Large \textbf{The genetic architecture of local adaptation I: The genomic landscape of 
foxtail pine (\textit{Pinus balfouriana} Grev. \& Balf.)}}
% Insert Author names, affiliations and corresponding author email.
\\
Christopher J.\ Friedline$^{1}$, 
Brandon M. Lind$^{1}$,
Erin M. Hobson,$^{1}$,
Douglas E. Harwood$^{1}$, 
Annette Delfino Mix$^{2}$,
Patricia E. Maloney$^{3}$, and
Andrew J. Eckert$^{1,4}$
\\
\bf{1} Department of Biology, Virginia Commonwealth University, Richmond, VA 23225
\\
\bf{2} Institute of Forest Genetics, USDA Pacific Southwest Research Station, Placerville, 
CA 95667
\\
\bf{3} Department of Plant Pathology, University of California, Davis, CA 95616
\\
\bf{4} Author for Correspondence

$\ast$ E-mail: aeckert2@vcu.edu
\end{flushleft}

\section{Abstract}

\section{Introduction}
The genetic architecture of fitness-related traits has been a major focus of geneticists for 
over a century. Genetic architecture refers to the number, type, effect size, genomic organization, 
interactions, and environmental dependency of the loci contributing to phenotypic variation 
which in turn creates variation in fitness among individuals within populations \citep{Eckert:2012a}. 
Interest in this architecture stems from the want to explain the nature of genetic variation which 
contributes to evolution via the accumulation of adaptations within lineages (i.e. adaptive evolution).
Evidence for adaptive evolution among populations of plants is commonly documented at both the phenotypic 
and molecular genetic levels \citep{Kawecki:2004, Pannell:2013}, so that some of the best
examples of adaptive evolution within lineages comes from the field of plant genetics.
Early efforts to understand the genetic architecture of fitness-related traits
focused primarily on the number and effect size of the loci underlying heritable, phenotypic variation \citep{Fisher:1930}. 
Recent work has extended this line of research, with myriad studies linking phenotypic with genetic variation 
through linkage mapping, both within pedigrees \citep{Mauricio:2001, Neale:2011, Ritland:2011} and within 
unstructured populations \citep{Ingvarsson:2011, Eckert:2013a}, 
or through quantitative genetic experimentation \citep{Anderson:2013a, Anderson:2013b, Fournier-Level:2013}. All of these approaches 
have been used extensively in forest genetics, with local adaptation among forest tree populations clearly
being documented over the past century \citep{White:2007, Neale:2011}. Despite great advances in experimental technology, empirical 
focus has remained almost fully on the number, effect size, type, and interactions among loci contributing 
to adaptive evolution \citep{Neale:2011, Alberto:2013}. The genomic organization of such loci, however, 
is also relevant, as it affects many of the other attributes of genetic architecture listed previously 
\citep{Kirkpatrick:2006, Yeaman:2011, Yeaman:2013}, including the ability to detect the loci as underlying trait variation. 
Any thorough examination of the genetic architecture of fitness-related traits, therefore, should include 
some aspect of the genomic organization of the loci contributing to trait variation. Here, we leverage 
this idea in the first of a series of papers dissecting the genetic architecture of fitness-related 
traits in a non-model conifer species, foxtail pine (\textit{Pinus balfouriana} Grev. \& Balf.), with the 
ultimate goal of testing explicit evolutionary hypotheses about the genomic organization of loci 
contributing to variation in fitness-related traits.

Ideally, the genomic organization of loci contributing to variation in fitness-related traits would follow 
naturally from the production of a genome sequence (i.e. a physical map). For many taxa, especially those with 
small to modest genome sizes, this is monetarily and computationally feasible using next-generation DNA sequencing 
technologies (Koboldt et al. 2013). For taxa with large or complex genomes, however, even the advent of next generation DNA 
sequencing does not solve the complexity and cost hurdles associated with the production of a finished genome sequence. Conifers have large and 
complex genomes (Murray 1998; Ahuja and Neale 2005), with estimated average genome sizes in \textit{Pinus} in the 
range of \SIrange{20}{30}{Gbp}. Several genome projects, each involving many laboratories, are underway or have been 
completed (Mackay et al. 2012; Nystedt et al. 2013). Even these efforts often initially result in limited information, 
as current assemblies of the Norway spruce (\textit{Picea abies} L.) and loblolly pine (\textit{Pinus taeda} L.) genomes 
contain millions of unordered contigs with average sizes in the thousands of base pairs (Nystedt et al. 2013; 
Neale, D.B. pers. comm.). An alternative, but not mutually exclusive, approach to describing the genome of an organism 
is that of linkage mapping. In this approach, genetic markers are ordered through observations of recombination events 
within pedigrees. This approach dates to the beginning of genetics and the logic has remained relatively unchanged 
since the first linkage maps were created in \textit{Drosophila} \citep{Sturtevant:1913}.

Renewed interest in linkage maps has occurred for two basic reasons. First, linkage maps can be used to order contigs 
created during genome sequencing projects (Mackay et al. 2012; Martinez-Garcia et al. 2013). In this fashion, linkage 
maps are used to help create larger contigs from those generated during the assembly. It is these larger contigs that 
create the utility that most practicing scientists attribute to genome sequences. Second, linkage maps are relatively 
easy to produce and provide a rich context with which to interpret population and quantitative genetic patterns of variation 
(e.g. Eckert et al. 2010a,b; Eckert et al. 2013; Yeaman 2013). They can also be used to test explicit hypotheses about 
the organization of loci contributing to adaptive evolution. For example, Yeaman and Whitlock (2011) developed theoretical 
predictions about the genomic organization of loci underlying patterns of local adaptation as a function of gene flow, 
so that loci contributing to local adaptation have differing spatial structure within genomes as a result of differing 
regimes of gene flow. The relevant scale (\textit{sensu} Houle et al. 2011) in these mathematical formulations is that 
of recombinational distance among loci, as it is recombination that breaks apart advantageous allelic combinations across 
loci, so that when matched with an appropriate study system, linkage maps provide the impetus to test basic evolutionary 
hypotheses. In this context, future additions of finished genome sequences will add only to the interpretation of 
results based on linkage maps.

Construction of linkage maps have a long history within forest genetics, mostly through their use in quantitative trait locus mapping (Ritland et al. 2011). 
Conifers in particular are highly amenable to linkage mapping, with approximately 25 different species currently having some 
form of linkage map (see Table 5-1 in Ritland et al. 2011). Much of the amenability of conifers to linkage mapping stems from 
the early establishment of breeding populations in economically important species and from the presence of a multicellular 
female gametophyte (i.e. the megagametophyte) from which the haploid product of maternal meiosis can be observed (Cairney and Pullman 2007; 
but see Pichot and El Maataoui 1997). Indeed, many of the first linkage maps in conifers were generated from collections of 
megagametophytes made from single trees (Tulsieram et al. 1992; Nelson et al. 1993; Kubisiak et al. 1996). Continued development 
of genetic marker technologies facilitated rapid development of linkage maps across a diversity of species, with the largest maps 
generated for economically important species (Ritland et al. 2011; e.g. Achere et al. 2004; Kang et al. 2010; Martinez-Garcia et al. 2013). 
The development of biologically informative markers in non-economically important conifers, however, was hampered by production cost of markers. 
For example, the vast majority of linkage maps outside of economically important species were created with uncharacterized genetic 
markers (e.g. Travis et al. 1998). The majority of this cost previously was in the two-step approach needed to generate biologically 
meaningful markers: polymorphism discovery via DNA sequencing of genic regions followed by genotyping of polymorphisms 
discovered in step one (cf. Eckert et al. 2013). As such, much of the knowledge about the genetic architecture of fitness-related 
traits outside of a handful of economically important conifer species was about the number and effect size of uncharacterized 
genetic markers (Ritland et al. 2011; see Nunes et al. 2012 for caveats). Cost restrictions, however, have largely disappeared 
as it is now feasible to jointly discover polymorphisms and genotype samples using high-throughput DNA sequencing approaches such as restriction-associated 
DNA sequencing (RADseq; e.g. Peterson et al. 2012). 

The generation of dense linkage maps from high-throughput RADseq data is a complex endeavor due to the inherent stochasticity
and error prone nature of the generation and analysis of short read data. Recent examples in several crop species highlight the
difficulties that must be overcome with respect to missing data and errors (Pfender et al. 2011; Ward et al. 2013). RADseq has been 
successively applied to samples taken from natural populations of non-model conifer species (Parchman et al. 2012), but has not yet 
been applied to linkage mapping in these species, so that an exploration of these methods to linkage mapping in the large and complex 
genomes of conifers is warranted. Here, we take this approach using megagametophyte arrays from four maternal trees of foxtail pine
to generate maternal linkage maps comprised of tens of thousands of markers. In doing so, we examine the inherent biases
to RADseq data generation in conifer genomes and how these biases affect downstream inferences of linkage maps. We subsequently discuss
the utility of our inferred linkage maps to future tests of how patterns of gene flow affect the genomic organization 
of loci underlying fitness-related traits in this and other conifer species characterized by strong patters of local adaptation.


\section{Materials and Methods}

\subsection{Focal species}
Foxtail pine is five needle species of \textit{Pinus} classified into 
subsection \textit{Balfourianae}, section \textit{Parrya}, and subgenus \textit{Strobus} 
(Gernandt et al. 2005). It is one of three species within subsection \textit{Balfourianae} 
(Bailey 1970) and generally is regarded as the sister species to Great Basin bristlecone pine (\textit{P. longaeva}; see 
Eckert and Hall 2006). The natural range of foxtail pine encompasses two 
regional populations located within California that are separated by approximately 500 km - 
the Klamath Mountains of northern California and the Sierra Nevada of southern California 
(Figure 1). These regional populations diverged approximately one million years ago (mya), 
with current levels of gene flow between these regional populations approximately zero 
(Eckert et al. 2008). Within each regional population, levels of genetic diversity and the 
degree of differentiation among local stands differ, with genetic diversity being highest in 
the southern Sierra Nevada population and genetic differentiation being the highest in the 
Klamath population (Oline et al. 2000; Eckert 2006; Eckert et al. 2008). The two regional 
populations of foxtail pine thus represent a powerful natural experiment within which to examine
the genomic organization of loci contributing to local adaptation.

\subsection{Sampling}
Seed lots from approximately 142 maternal trees distributed throughout the natural range 
of foxtail pine were obtained during 2011 and 2012. Of these 142 maternal trees, 73 were sampled from the 
Klamath region, while 69 were sampled from the southern Sierra Nevada region. Approximately 
50 seeds were germinated from each seed lot and 35 of those 50 seedlings were planted in a 
common garden located at the USDA Institute of Forest Genetics, Placerville, California. The 
common garden was established using a randomized block design.
Four of the 140 maternal trees were selected at random ($n = 3$ from the Klamath region and $n = 2$ from 
the southern Sierra Nevada) for linkage analysis. For each of these trees, \SIrange{75}{100}{} 
seeds were germinated and planted in the common garden. Upon germination, haploid 
megagametophyte tissue was rescued from each growing seedling, cleaned, and stored for further 
analysis in \SI{1.5}{\mL} Eppendorf tubes at \SI{-20}{\celsius}.


\subsection{Library Preparation and Sequencing}
Total genomic DNA was isolated from each rescued megagametophyte using the DNeasy 96 Plant 
kit following the manufacturer’s protocol (Qiagen, Germantown, MD). RADseq (Davey and Blaxter 2010; Parchman et al. 2012; Peterson et al. 2012) 
was used to generate a genome-wide set of 
single nucleotide polymorphism (SNP) markers for linkage mapping following the protocol 
outlined by Parchman et al. (2012). In brief, this protocol is a double digestion RADseq 
approach based on digestion of total genomic DNA using EcoR1 and MseI followed by single-end 
sequencing on the Illumina HiSeq platform. Following digestion, adaptors 
containing amplification and sequencing primers, as well as barcodes for multiplexing, 
were ligated on to digested DNA fragments. We chose to multiplex 96 samples using the 
barcodes available from Parchman et al. (2012). These barcodes are a mixture of 8, 9, and 
10 bp tags that differ by at least four bases, so as to accommodate sample identification in the 
presence of sequencing errors. Following ligation, successfully ligated DNA fragments were 
amplified using PCR and amplified fragments were size selected using gel electrophoresis. We selected 
fragments in the size range of 400 bp (\SIrange{300}{500}{bp}) by excising and purifying pooled DNA from 2.5\% 
agarose gels using QIAquick Gel Extraction Kits (Qiagen). Further details, including relevant reagents and 
oligonucleotide sequences, can be found in File S1. All DNA sequencing was performed on the Illumina HiSeq 2000 or 2500
platform at the VCU Nucleic Acids Research Facility (\url{http://www.narf.vcu.edu/}).


\subsection{DNA Sequence Analysis}

DNA sequences (reads) were demultiplexed into sample-level fastq files, following quality control 
and filtering.  The filtering pipeline was adapted from (cf ocean paper ref), and is briefly: reads 
containing any N beyond the first base were excluded. Reads having N as the first base were shifted 
to exclude it.  Additional quality filtering ensured that all reads in the resulting set for downstream 
processing had a minimum average quality score of 30 over 5 base-pair (bp), overlapping windows 
and that not more than \SI{20}{\percent} of the bases had quality scores below 30. Reads passing the 
quality control steps were demultiplexed into sample-specific fastq files by exact pattern matching to 
known barcodes.

From each set of demultiplexed reads, the individual with the largest number of reads as assembled using 
Velvet (Zerbino, version 1.2.10), with hash length ($k$) coverage cutoff optimized using parameter sweeps of $k$ 
through the contributed VelvetOptimiser (\url{http://www.vicbioinformatics.com}, version 2.2.4) 
script (for odd $k$ on $k=[19,65]$).  Each assembly was indexed and aligned to the original set of reads 
using Bowtie2 (citation) (\texttt{--local --very-sensitive}).  Additionally, assemblies were evaluated in a 
likelihood framework using LAP \citep{Ghodsi:2013bc}

\subsection{Linkage Analysis}

\section{Results}

\section{Discussion}

\section{Acknowledgements}

The authors would like to thank the staff at the USDA Institute of Forest Genetics, the 
VCU Nucleic Acids Research Facility, and the VCU Center for High Performance Computing. 
In addition, we would like to thank Tom Blush and Tom Burt for help in obtaining 
seeds. Funding for this project was made available to AJE via startup funds from Virginia 
Commonwealth University. CJF was supported by the National Science Foundation (NSF) National Plant Genome 
Initiative (NPGI): Postdoctoral Research Fellowship in Biology (PRFB) FY 2013 Award \#NSF-NPGI-PRFB-1306622.

\printbibliography
\end{document}
